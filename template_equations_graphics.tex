% !TeX encoding = UTF-8


\documentclass[12pt]{article}  %  Класса article хватит для курсовой и диплома 

\usepackage[utf8x]{inputenc}   % Подключение пакета inputenc и задание кодировки
\usepackage[english, russian]{babel}    % Локализация 
\usepackage{amsmath}		   % Пакет специальных символов
\usepackage{amsfonts}		   
\usepackage{amssymb}		   % Этих пакетов хватает для набора большинства спецсимволов	
\usepackage{graphicx}          % Графика
\usepackage{hyperref} 		   % Навигация по ссылкам в документе
\usepackage{cite} 		       % Продвинутое цитирование
\usepackage[a4paper, top=25mm, left=30mm, right=10mm, bottom=25mm]{geometry} % Поля 








\begin{document}         

%---------------------------%
% 	  "Сложные" формулы	    %
%					        %

\section{<<Сложные>> формулы}

\begin{align}
&\|A\| \equiv \sup \limits_{x\neq 0} \left\{ \frac{\|Ax\|}{\|x\|} \right\}
\text{ --- норма оператора } A  \label{operator_norm}  \\ 
&\|f\|_2 = \left( \int\limits_a^b f^2(x)  dx \right)^{1/2} \notag \text{ ~--- норма в $L^2(a, b)$ }
\end{align}
\begin{equation} \label{eq5243}
a = b + c
\end{equation}

\begin{equation} \label{eq1}
a = b + c
\end{equation}

согласно \ref{eq1}, 




%---------------------------%
% 	  "Длинные" формулы	    %
%					        %

\section{``Длинные'' формулы}

\begin{multline*} 
\int\limits_0^1 x \, dx = \int\limits_0^1 \biggl( \frac{x^2}{2} \biggr)' \, dx = 
\left. \frac{x^2}{2}\right|_0^1 = \frac{1}{2} = \frac{1}{2} \cdot 1 = 
\frac{1}{2} \cdot \left(\frac{1}{2} + \frac{1}{4} + \frac{1}{8} + \cdots \ \right) =  \\
= \frac{1}{2} \cdot \left( \sum_{1}^{\infty} \frac{1}{2^i} \right) = 
\frac{1}{2} \cdot \left(\frac{1}{2} + \frac{1}{4} +  \frac{1}{8} + \cdots \ \right) = 
\frac{1}{2} \cdot 1 = \frac{1}{2} = \left. \frac{x^2}{2}\right|_0^1 = 
\int\limits_0^1 \biggl( \frac{x^2}{2} \biggr)' \, dx = \int\limits_0^1 x \, dx
\end{multline*}

\begin{equation}
	\begin{split}
		\int\limits_0^1 x \, dx 
		&= \int\limits_0^1 \biggl( \frac{x^2}{2} \biggr)' \, dx = 
		\left. \frac{x^2}{2}\right|_0^1 = \frac{1}{2} = \frac{1}{2} \cdot 1 = \\
		&= \frac{1}{2} \cdot \left(\frac{1}{2} + \frac{1}{4} + \frac{1}{8} + \cdots \ \right) = \frac{1}{2} \cdot \left( \sum_{1}^{\infty} \frac{1}{2^i} \right)
	\end{split}
\end{equation}





%---------------------------%
% 	     Матрица     	    %
%					        %


%\section{Матрица}
%\begin{equation}
%	\begin{pmatrix}
%		a_{11} & \dots & a_{1n}  \\
%		\vdots &  & \vdots       \\
%		a_{m1} & \dots & a_{mn}
%	\end{pmatrix}
%\end{equation}





%---------------------------%
% 	     Рисунок     	    %
%					        %

%\section{Рисунок}
%
%\begin{figure}[h]   %  расположение на странице
%	\begin{center}
%		\includegraphics[keepaspectratio,width=160mm]{pic.pdf}
%	\end{center}
%	\caption{Важный рисунок}
%	\label{important_pic}
%\end{figure}
%
%Изображение на рис. \ref{important_pic} очень характерно для статей.




\end{document}
