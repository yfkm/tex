% !TeX encoding = UTF-8


\documentclass[12pt]{article}  %  Класса article хватит для курсовой и диплома 

\usepackage[utf8x]{inputenc}   % Подключение пакета inputenc и задание кодировки
\usepackage[english, russian]{babel}    % Локализация 
\usepackage{amsmath}		   % Пакет специальных символов
\usepackage{amsfonts}		   
\usepackage{amssymb}		   % Этих пакетов хватает для набора большинства спецсимволов	
\usepackage{graphicx}          % Графика
\usepackage{hyperref} 		   % Навигация по ссылкам в документе
\usepackage{cite} 		       % Продвинутое цитирование
\usepackage[a4paper, top=25mm, left=30mm, right=10mm, bottom=25mm]{geometry} % Поля 


\begin{document}         

Всюду далее будем считать, что, как и в [4], [5], $N_1 = N_2 = 4$. 
Для исследования локальной динамикии системы (12) \-- (7) в окрестности нулевого решения при фиксированном $\nu > 0$ и достаточн малом $0 < \varepsilon \ll 1$ выполним стандартную замену метода нормальных форм (см. [9])

\begin{equation}
u_{k,j}(t,\tau,\varepsilon) = u_0(t,\tau,k,j) + \varepsilon u_1(t,\tau,k,j) + \dots,\ \ \ \tau = \varepsilon t, \ \ \ k,j = 1,\dots,4.
\end{equation}
Здесь
\begin{equation}
u_0(t,\tau,k,j) = \sum_{n,m=1}^{4}\left[z_{n,m}(\tau)\exp(i\omega_{n,m}t)+\bar{z}_{n,m}(\tau)\exp(-i\omega_{n,m}t)\right]e_{n,m}(k,j),
\end{equation}
причем $z_{n,m}$ \--- пока неизвестные подлежащие определению амплитуды.
 
В данной работе обратимся лишь к нерезонансному случаю, т.е. выберем параметры $\varsigma_1, \varsigma_2$ таким образом, чтобы исключить всевозможные резонансы третьего порядка между собственными частотами $\omega_{n,m}$, иначк говоря, исключим одновременное выполнение равенств вида $\omega_{n_0,m_0} = s_1\omega_{n_1,m_1} +  s_2\omega_{n_2,m_2} + s_3\omega_{n_3,m_3},\ n_0 = \pm n_1 \pm n_2 \pm n_3, m_0 = \pm m_1 \pm m_2 \pm m_3$ для любых наборов индексов  $n_k,m_k, k=0,\dots,3$, таких что $(s_1, s_2, s_3): |s_1|+|s_2|+|s_3| = 3$ и при любой расстановке знаков во втором и третьем соотношениях. Тождественные резонансы $\omega_{n_0,m_0}=\omega_{n_0,m_0}+\omega_{n_1,m_1}-\omega_{n_1,m_1}$ из рассмотрения, естесственно, исключаются.

После подстановки (1) в краевую задачу (12) \-- (7) приравниваем коэффициенты при одинаковых степенях $\varepsilon$, после чего на первом шаге получаем верное тождество, а на втором приходим к следующей краевой задаче для $u_1(t,\tau,k,j):$
\begin{multline}
\frac{\partial^2u_1}{\partial t^2}+u_1-\mathcal{L}^{*}u_1 = \frac{\partial}{\partial t}\left[u_0+\nu \mathcal{L}^{*}u_0-\frac{u_0^3}{3}-2\frac{\partial u_0}{\partial\tau}\right],\\
u_1(0,j)=u_1(1,j), u_1(4,j)=u_1(5,j), u_1(k,0)=u_1(k,1), u_1(k,4)=u_1(k,5), \\
k=1,\dots,4, j=1,\dots,4.
\end{multline}
Система (3) разрешива в классе периодических по $t$ функций в том и только в том случае, если в неоднородности отсутствуют гармоники вида (13). Приравнивая коэффициенты при этих гармониках к нулю, приходим к системе относительно неизветсных комплексных амплитуд $z_{n,m}$
\begin{multline}
2\frac{dz{n,m}}{d\tau}=(1-\nu(\omega_{n,m}^2-1))z_{n,m}-\frac94 z_{n,m}|z_{n,m}|^2 - 3\sum_{k=0,k\not=n}^{3}z_{n,m}|z_{k,m}|^2 - \\
-3\sum_{s=0,s\not=m}^{3}z_{n,m}|z_{n,s}|^2-2\sum_{k,s=0,k\not=n,s\not=m}^{3}z_{n,m}|z_{k,s}|^2, \ \ \ n=0,\dots,3, \ m=0,\dots,3.
\end{multline}
Воспользовавшись заменой $z_{n,m}=\rho_{n,m}\exp(i\varPhi_{n,m})$, приходим к следующей системе, с которой и будем иметь дело в дальнейшем:
\begin{multline}
\frac{d\rho_{n,m}}{d\tau}=(1-\nu(\omega_{n,m}^2-1))\rho_{n,m}-\frac94 \rho_{n,m}^3 - 3\rho_{n,m}\sum_{k=0,k\not=n}^{3}\rho_{k,m}^2 - \\
-3\rho_{n,m}\sum_{s=0,s\not=m}^{3}\rho_{n,s}^2-2\rho_{n,m}\sum_{k,s=0,k\not=n,s\not=m}^{3}\rho_{k,s}^2,\ \ \ n=0,\dots,3, \ m=0,\dots,3.
\end{multline}

В статье [2] система (5) рассмотрена в бесконечномерном случае, как оказалось, исследование ее нулевого состояния равновесия и однокомпонентных режимов можно выполнить по аналогии с [2]. Задача поиска условий существования и устойчивости многокомпонентных состоянийй равновесия в данном случае имеет существенную специфику, поэтому остановимся именно на ней. 

Сначала рассмотрим режимы с двумя ненулевыми компонентами $\rho_{p,q} \not=0, \rho_{r,s}\not=0, \rho_{i,j}=0, i,j=0,\dots,3, \ \ i\not=p \bigvee j\not=q, \ \ i\not=r \bigvee j\not=s$. Имеем 
\begin{equation}
\begin{split}
&\rho_{p,q}^2 = \frac{4(4C_{(p,q)(r,s)}(1-\nu(\omega_{r,s}^2-1))-9(1-\nu(\omega_{p,q}^2-1)))}{16C^2_{(p,q)(r,s)}-81},\\
&\rho_{r,s}^2 = \frac{4(4C_{(p,q)(r,s)}(1-\nu(\omega_{p,q}^2-1))-9(1-\nu(\omega_{r,s}^2-1)))}{16C^2_{(p,q)(r,s)}-81},\\
&C_{(p,q)(r,s)}=3, \ \ p=r \bigvee q=s, \ \  C_{(p,q)(r,s)}=2, \ \ p\not=r \bigwedge  q\not=s.
\end{split}
\end{equation}

Линеаризуя систему (5) на состоянии равновесия (6), получаем $16\times16$ матрицу с коэффициентами
\begin{equation}
\begin{split}
&\alpha_{(p,q),(p,q)}=-\frac92\rho^2_{p,q}, \ \ \alpha_{(r,s),(r,s)}=-\frac92\rho^2_{r,s},\\
&\alpha_{(p,q),(r,s)}=-2C_{(p,q)(r,s)}\rho_{p,q}\rho_{r,s}, \ \ \alpha_{(r,s),(p,q)}=-2C_{(p,q)(r,s)}\rho_{p,q}\rho_{r,s},\\
&\alpha_{(i,j)(i,j)} = (1 - \nu(\omega^2_{i,j}-1)) - C_{(p,q)(i,j)}\rho^2_{p,q} - C_{(r,s)(i,j)}\rho^2_{r,s},\\
&i\not=p \bigvee j\not=q, \ i\not=r \bigvee j\not=s,\\
&\alpha_{(i,j)(k,l)}=0,\\
&i\not=p \bigvee j\not=q, \ i\not=r \bigvee j\not=s, \ \ i\not=k, j\not=l,\\
&i=0,\dots,3,j=0,\dots,3.
\end{split}
\end{equation}
Анализ собственных значений этой матрицы приводит к следующему условию устойчивости состояний равновесия (6) :
\begin{multline}
(1 - \nu(\omega^2_{i,j}-1)) < \frac8{17+4(C_{(p,q)(r,s)}-2)}((1 - \nu(\omega^2_{p,q}-1))+(1 - \nu(\omega^2_{r,s}-1))),\\
\forall i\not=p\bigvee j\not=q, \ \ i\not=r\bigvee j\not=s, \ \ i=0,\dots,3,j=0,\dots,3,
\end{multline}
из которого нетрудно получить условие сосуществования максимального числа таких устойчивых режимов
\begin{equation}
(1 - \nu(\omega^2_{3,3}-1)) < \frac{16}{21}(1 - \nu(\omega^2_{0,0}-1))
\end{equation}
\newtheorem{Th}{Теорема}
\begin{Th}
	Пусть для краевой задачи (12) \--- (7) выполнено условие (23), где $\omega_{0,0},\omega{3,3}$ задаются равенством (14), тогда существоет такое $\varepsilon_0>0$, что для всех $0<\varepsilon<\varepsilon_0$ задача (12) \--- (7) имеет ровно 120 орбитально асимптотически устойчивых торов, асимптотика которых задается следующей формулой
\end{Th}
      
\begin{multline}
u(t,k,j) = 2\sqrt{\frac{4C_{(p,q)(r,s)}(1-\nu(\omega^2_{r,s}-1))-9(1-\nu(\omega^2_{p,q}-1))}{16C^2_{(p,q)(r,s)}-81}}\cos(\omega_{p,q}t+\varphi_{p,q})\times\\
\times\cos\frac{\pi p(k-\frac{1}{2})}4\cos\frac{\pi q(j-\frac{1}{2})}4+2\sqrt{\frac{4C_{(r,s)(p,q)}(1-\nu(\omega^2_{p,q}-1))-9(1-\nu(\omega^2_{r,s}-1))}{16C^2_{(r,s)(p,q)}-81}}\times\\
\times\cos(\omega_{r,s}t+\varphi_{r,s})\cos\frac{\pi r(k-\frac{1}{2})}4\cos\frac{\pi s(j-\frac{1}{2})}4 + O(\varepsilon),\\
p=0,\dots,3, \ \ \ q=0,\dots,3, \ \ \ r=0,\dots,3, \ \ \ s=0,\dots,3.
\end{multline}

Обратимся теперь к случаю,  когда число ненулевых компонент в состоянии равновесия превышает 2:$\rho_{p_1,q_1}\not=0,\dots,\rho_{p_m,q_m}\not=0, m\geq3$. Компоненты матрицы устойчивости имеют в этом случае вид
\begin{equation}
\begin{split}
&\alpha_{(p_1,q_1),(p_1,q_1)}=-\frac{9}{2}\rho^2_{p_1,q_1}, \ \ \ \alpha_{(p_2,q_2),(p_2,q_2)}=-\frac{9}{2}\rho^2_{p_2,q_2}, \\
&\dots,\\
&\alpha_{(p_1,q_1),(p_2,q_2)}=-2C_{(p_1,q_1),(p_2,q_2)}\rho_{p_1,q_1}\rho_{p_2,q_2}, \  \alpha_{(p_2,q_2),(p_1,q_1)}=-2C_{(p_2,q_2),(p_1,q_1)}\rho_{p_2,q_2}\rho_{p_1,q_1}, \\
&\dots,\\
&\alpha_{(p_1,q_1),(i,j)} = 0,\ \ \ alpha_{(p_2,q_2),(i,j)} = 0,\\
&\dots,\\
&\alpha{(i,j),(i,j)}=(1-\nu(\omega^2_{i,j}))-\sum_{s=1}^{m}C_{(p_s,q_s)(i,j)}\rho^2_{p_s,q_s},\\
&i\not=p_s \bigvee j\not=q_s, s=1,\dots,m\\
&\alpha_{(i,j),(l,k)} = 0, \\
&i\not=p_s \bigvee j\not=q_s, s=1,\dots,m \ \ \ i\not=k,j\not=l\\
&i=0,\dots,3,j=0,\dots,3.
\end{split}
\end{equation}

В случае, если хотя бы одно из чисел $C_{(p_s,q_s),(p_w,q_w)}, \ \ s=1,\dots,m, \ \ w=1,\dots,m, s\not=m$ равно трем,коэффициенты характеристического уравнения, полученного из (25), имеют разные знаки, что гарантирует неустойчивость соответствующего состояния равновесия. Пользуясь этим условием, заключаем, что неустойчивыми будут все состояния равновесия с числом ненулевых компонент большим либо равным 5. Тем самым, остается рассмотреть случаи $m=3$ и $m=4$. Для $m=3$ также оказывается, что коэффициенты характеристического многочлена имеют разные знаки, что естественно влечет неустойчивость.
	
Случай $m=4$ несколько более сложный. Выпишем значения ненулевых компонентов такого состояния равновесия
\begin{equation}
\begin{split}
&\rho^2_{p_1,q_1}=\frac{1}{33}(100(1-\nu(\omega^2_{p_1,q_1}-1))-32(1-\nu(\omega^2_{p_2,q_2}-1))-\\
&-32(1-\nu(\omega^2_{p_3,q_3}-1))-32(1-\nu(\omega^2_{p_4,q_4}-1))),\\
&\rho^2_{p_2,q_2}=\frac{1}{33}(100(1-\nu(\omega^2_{p_2,q_2}-1))-32(1-\nu(\omega^2_{p_1,q_1}-1))-\\
&-32(1-\nu(\omega^2_{p_3,q_3}-1))-32(1-\nu(\omega^2_{p_4,q_4}-1))),\\
&\rho^2_{p_3,q_3}=\frac{1}{33}(100(1-\nu(\omega^2_{p_3,q_3}-1))-32(1-\nu(\omega^2_{p_2,q_2}-1))-\\
&-32(1-\nu(\omega^2_{p_1,q_1}-1))-32(1-\nu(\omega^2_{p_4,q_4}-1))),\\
&\rho^2_{p_4,q_4}=\frac{1}{33}(100(1-\nu(\omega^2_{p_4,q_4}-1))-32(1-\nu(\omega^2_{p_2,q_2}-1))-\\
&-32(1-\nu(\omega^2_{p_3,q_3}-1))-32(1-\nu(\omega^2_{p_1,q_1}-1))).
\end{split}
\end{equation}
Понятно, что условия существования таких состояний равновесия имеют вид
\begin{equation}
\begin{split}
&100(1-\nu(\omega^2_{p_1,q_1}-1))-32(1-\nu(\omega^2_{p_2,q_2}-1))-\\
&-32(1-\nu(\omega^2_{p_3,q_3}-1))-32(1-\nu(\omega^2_{p_4,q_4}-1)) > 0, \\
&100(1-\nu(\omega^2_{p_2,q_2}-1))-32(1-\nu(\omega^2_{p_1,q_1}-1))-\\
&-32(1-\nu(\omega^2_{p_3,q_3}-1))-32(1-\nu(\omega^2_{p_4,q_4}-1)) > 0, \\
&100(1-\nu(\omega^2_{p_3,q_3}-1))-32(1-\nu(\omega^2_{p_2,q_2}-1))-\\
&-32(1-\nu(\omega^2_{p_1,q_1}-1))-32(1-\nu(\omega^2_{p_4,q_4}-1)) > 0, \\
&100(1-\nu(\omega^2_{p_4,q_4}-1))-32(1-\nu(\omega^2_{p_2,q_2}-1))-\\
&-32(1-\nu(\omega^2_{p_3,q_3}-1))-32(1-\nu(\omega^2_{p_1,q_1}-1)) > 0, \\
\end{split}
\end{equation}
Задача об устойчивости данных состояний равновесия распадается на две независящие друг от друга части. Коэффициенты первой из них имеют вид
\begin{equation}
\begin{split}
&\alpha_{(p_1,q_1),(p_1,q_1)} = -\frac{9}{2}\rho^2_{p_1,q_1}, \ \ \alpha_{(p_2,q_2),(p_2,q_2)} = -\frac{9}{2}\rho^2_{p_2,q_2},\\
&\dots,\\
&\alpha_{(p_1,q_1),(p_2,q_2)} = - 2C_{(p_1,q_1)(p_2,q_2)}\rho^2_{p_1,q_1}\rho^2_{p_2,q_2}, \ \ \alpha_{(p_2,q_2),(p_1,q_1)} = - 2C_{(p_2,q_2)(p_1,q_1)}\rho^2_{p_2,q_2}\rho^2_{p_1,q_1},\\
&\dots
\end{split}
\end{equation}
а второй \---
\begin{equation}
\begin{split}
&\alpha_{(i,j),(i,j)} = (1-\nu(\omega^2_{i,j}-1)) - \sum_{s=1}^{m}C_{(p_s,q_s)(i,j)}\rho^2_{(p_s,q_s)}, \\
&i\not=p_s \bigvee j\not=q_s, s=1,\dots,4,\\
&\alpha_{(i,j),(k,l)}=0,\\
&i\not=p_s \bigvee j\not=q_s, s=1,\dots,4, \ \ i\not=k, j\not=l\\
&i=0,\dots,3,j=0,\dots,3.
\end{split}
\end{equation}

Все коэффициенты характеристического полинома первой из матриц положительны и имеют вид
\begin{equation*}
\begin{split}
&A_0=1; \ \ A_1=\frac92(p_1^2+p_2^2+p_3^2+p_4^2);\\
&A_2=\frac{17}4(p_1^2p_2^2+p_1^2p_3^2+p_1^2p_4^2+p_2^2p_3^2+p_2^2p_4^2+p_3^2p_4^2);\\
&A_3=\frac{25}8(p_1^2p_2^2p_3^2+p_1^2p_2^2p_4^2+p_1^2p_3^2p_4^2+p_2^2p_3^2p_4^2);\\
&A_4=\frac{33}{16}(p_1^2p_2^2p_3^2p_4^2),\\
\end{split}
\end{equation*}
позволяющий утверждать, что все корни данного многочлена имеют отрицательную действительные части.

Необходимым и достаточным условием устойчивости второй матрицы являются неравенства
\begin{multline}
(1-\nu(\omega^2_{i,j}-1))<\frac{8}{33}(1-\nu(\omega^2_{p_1,q_1}-1))+(1-\nu(\omega^2_{p_2,q_2}-1))+\\
+(1-\nu(\omega^2_{p_3,q_3}-1))+(1-\nu(\omega^2_{p_4,q_4}-1)),\\
\forall i\not=p_s \bigvee j\not=q_s,\ \ s=1,\dots,4, \ \ \ i=0,\dots,3,j=0,\dots,3,
\end{multline}
причем при выполнении условия
\begin{equation}
(1-\nu(\omega^2_{3,3}-1))<\frac{32}{33}(1-\nu(\omega^2_{0,0}-1))
\end{equation}
в системе (19) существует максимальное количество описанных выше четырехкомпонентных режимов.

Подведя итог исслендования состояний равновесия системы (19) в нерезонансном случае, можно сформулировать дополнительно к теореме 1 следующие два утверждения.

\begin{Th}
	Пусть для краевой задачи (12) \--- (7) выполнено условие (30), где $\omega_{0,0},\omega{3,3}$ задаются равенством (14), тогда существоет такое $\varepsilon_0>0$, что для всех $0<\varepsilon<\varepsilon_0$ задача (12) \--- (7) имеет ровно 24 орбитально асимптотически устойчивых торов, асимптотика которых задается следующей формулой:
\end{Th}
\begin{multline}
u(t,k,j)=\rho_{q_1,p_1}\cos(\omega_{p_1,q_1}t + \varphi_{p_1,q_1})\cos\frac{\pi p_1(k-\frac12)}4\cos\frac{\pi q_1(j-\frac12)}{4}\\
\rho_{q_2,p_2}\cos(\omega_{p_2,q_2}t + \varphi_{p_2,q_2})\cos\frac{\pi p_2(k-\frac12)}4\cos\frac{\pi q_2(j-\frac12)}{4}\\
\rho_{q_3,p_3}\cos(\omega_{p_3,q_3}t + \varphi_{p_3,q_3})\cos\frac{\pi p_3(k-\frac12)}4\cos\frac{\pi q_3(j-\frac12)}{4}\\
\rho_{q_4,p_4}\cos(\omega_{p_4,q_4}t + \varphi_{p_4,q_4})\cos\frac{\pi p_4(k-\frac12)}4\cos\frac{\pi q_4(j-\frac12)}{4} + O(\varepsilon),\\
p_s=1,\dots,4,\ \ q_s=1,\dots,4,\ \ s=1,\dots,4.\ \ 
\end{multline}
\begin{Th}
	Все состояния равновесия системы (19) с числом ненулевых компонент равным трем, а также больше либо равным 5 неусточивы.
\end{Th}
\section{Решетка Скотта с краевыми условиями Дирихле}
Рассмотрим систему (12) - (6). Для нее справедливы следующие рассуждения, аналогичные тем, что приведены в предыдущем разделе, поэтому большую часть их мы опустим и сруау перейдем к формулировке результатов.

Сразу же отметим, что собственные векторы системы (12-5) имеют вид
\begin{equation}
\begin{split}
u_{(m,n)}=\exp(\pm i\omega_{n,m})e_{n,m}(k,j), \ \ e_{n,m}(k,j) = 2\sin\frac{\pi nk}{5}\sin\frac{\pi mk}{5},&\\
n=1,\dots,4, \ \ m=1,\dots,4, \ \ k=1,\dots,4, \ \ j=1,\dots,4&
\end{split}
\end{equation}
где
\begin{equation}
\omega_{n,m}=\sqrt{1+2\delta^2_1\left(1-\cos\frac{\pi n}5\right)+\delta^2_2\left(1-\cos\frac{\pi n}5\right)}, n=1,\dots,4, m=1,\dots,4,
\end{equation}

Нормальная форма в нерезонансном случае выписывается следующим образом:
\begin{multline}
2\frac{dz_{n,m}}{dr}=(1-\nu(\omega^2_{n,m}-1))z_{n,m} - \frac94)z_{n,m}|z_{n,m}|^2-3\sum_{k=1,k\not=n}^{4}z_{n,m}|z_{k,m}|^2-\\
	-3\sum_{s=1,s\not=m}^{4}z_{n,m}|z_{n,s}|^2 -2\sum_{k,s=1,s\not=m,k\not=n}^{4}z_{n,m}|z_{k,s}|^2, \ \ n=1,\dots,4, \ \ m=0,\dots,3
\end{multline}

Анализ этой системы позволяет сделать вывод о динамике системы (12) - (6) и сформулировать для ее условия существования ассимптотически орбитально устойчивых многочастотных режимов. В частности для двухчастотных режимов это условие имеет вид
\begin{multline}
(1-\nu(\omega^2_{i,j}-1))<\frac8{17+4C_{(p,q)(r,s)}-2}(1-\nu(\omega^2_{p,q}-1))+(1-\nu(\omega^2_{r,s}-1)),\\
\forall i\not=p \bigvee j\not=q, \  i\not=r \bigvee j\not=s, \ \ i,j=0,\dots,3,
\end{multline}
откуда нетрудно получить условие сосуществования максимального числа таких устойчивых режимов
\begin{equation}
(1-\nu(\omega^2_{4,4}-1))<\frac{16}{21}(1-\nu(\omega^2_{1,1}-1)).
\end{equation}

Сформулируем основные утвержднеиня, выполненные для данного случая.
\begin{Th}
	Пусть для краевой задачи (12) \--- (6) выполнено условие (36), где $\omega_{0,0},\omega{3,3}$ задаются равенством (14), тогда существоет такое $\varepsilon_0>0$, что для всех $0<\varepsilon<\varepsilon_0$ задача (12) \--- (7) имеет ровно 120 орбитально асимптотически устойчивых торов, асимптотика которых задается следующей формулой:
\end{Th}
\begin{multline}
u(t,k,j) = 2\sqrt{\frac{4C_{(p,q)(r,s)}(1-\nu(\omega^2_{r,s}-1))-9(1-\nu(\omega^2_{p,q}-1))}{16C^2_{(p,q)(r,s)}-81}}\cos(\omega_{p,q}t+\varphi_{p,q})\times\\
\times\sin\frac{\pi pk}5\sin\frac{\pi qj}5 +2\sqrt{\frac{4C_{(r,s)(p,q)}(1-\nu(\omega^2_{p,q}-1))-9(1-\nu(\omega^2_{r,s}-1))}{16C^2_{(r,s)(p,q)}-81}}\times\\
\times\cos(\omega_{r,s}t+\varphi_{r,s})\sin\frac{\pi rk}4\sin\frac{\pi sj}4 + O(\varepsilon),\\
p,q,r,s=0,\dots,3, \ \  k,j=1,\dots,4.
\end{multline}

Для четырехкомпонентных режимов условие устойчивости соответствующего состояния равновесия может быть сформулировано следующим образом:
\begin{equation}
\begin{split}
&(1-\nu(\omega^2_{i,j}-1))<\frac{8}{33}(1-\nu(\omega^2_{p_1,q_1}-1))+(1-\nu(\omega^2_{p_2,q_2}-1))+\\
&+(1-\nu(\omega^2_{p_3,q_3}-1))+(1-\nu(\omega^2_{p_4,q_4}-1)),\\
&\forall i\not=p_s \bigvee j\not=q_s,\ \ s=1,\dots,4, \\ 
&i,j=0,\dots,3,
\end{split}
\end{equation}
а при выполнении условия 
\begin{equation}
(1-\nu(\omega^2_{3,3}-1))<\frac{32}{33}(1-\nu(\omega^2_{0,0}-1))
\end{equation}
в исходной системе сосуществует максимальное число четырехкомпонентных режимов, в частности выполнена теорема
\begin{Th}
	Пусть для краевой задачи (12) \--- (6) выполнено условие (36), где $\omega_{0,0},\omega{3,3}$ задаются равенством (14), тогда существоет такое $\varepsilon_0>0$, что для всех $0<\varepsilon<\varepsilon_0$ задача (12) \--- (7) имеет ровно 24 орбитально асимптотически устойчивых торов, асимптотика которых задается следующей формулой:
\end{Th}

\begin{equation}
\begin{split}
&u(t,k,j)=\rho_{q_1,p_1}\cos(\omega_{p_1,q_1}t + \varphi_{p_1,q_1})\sin\frac{\pi p_1k}5\sin\frac{\pi q_1j}{5}\\
&\rho_{q_2,p_2}\cos(\omega_{p_2,q_2}t + \varphi_{p_2,q_2})\sin\frac{\pi p_2k}5\sin\frac{\pi q_2j}5\\
&\rho_{q_3,p_3}\cos(\omega_{p_3,q_3}t + \varphi_{p_3,q_3})\sin\frac{\pi p_3k}5\sin\frac{\pi q_3j}5\\
&\rho_{q_4,p_4}\cos(\omega_{p_4,q_4}t + \varphi_{p_4,q_4})\sin\frac{\pi p_4k}5\sin\frac{\pi q_4j}5 + O(\varepsilon),\\
&p_s=1,\dots,4,\ \ q_s=1,\dots,4,\ \ s=1,\dots,4.\ \ 
\end{split}
\end{equation}

В заключение заметим, что обе системы (12) - (7) и (12) - (6) демонстрируют достаточно сложную динамику, состоящую в сосуществовании большого числа многочастотных режимов. Тем самым здесь реализуется аналог хаотической буферности, найденной в [2], сопровождающийся, однако, рядом эффектов, обусловленных конечномерностью задачи.
\end{document}
