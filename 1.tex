% !TeX encoding = UTF-8


\documentclass[12pt]{article}  %  Класса article хватит для курсовой и диплома 

\usepackage[utf8x]{inputenc}   % Подключение пакета inputenc и задание кодировки
\usepackage[english, russian]{babel}    % Локализация 
\usepackage{amsmath}		   % Пакет специальных символов
\usepackage{amsfonts}		   
\usepackage{amssymb}		   % Этих пакетов хватает для набора большинства спецсимволов	
\usepackage{graphicx}          % Графика
\usepackage{hyperref} 		   % Навигация по ссылкам в документе
\usepackage{cite} 		       % Продвинутое цитирование
\usepackage[a4paper, top=25mm, left=30mm, right=10mm, bottom=25mm]{geometry} % Поля 








\begin{document}         

Всюду далее будем считать, что, как и в [4], [5], $N_1 = N_2 = 4$. 
Для исследования локальной динамикии системы (12) \-- (7) в окрестности нулевого решения при фиксированном $\nu > 0$ и достаточн малом $0 < \varepsilon \ll 1$ выполним стандартную замену метода нормальных форм (см. [9])

\begin{equation}
u_{k,j}(t,\tau,\varepsilon) = u_0(t,\tau,k,j) + \varepsilon u_1(t,\tau,k,j) + \dots,\ \ \ \tau = \varepsilon t, \ \ \ k,j = 1,\dots,4.
\end{equation}
Здесь
\begin{equation}
u_0(t,\tau,k,j) = \sum_{n,m=1}^{4}\left[z_{n,m}(\tau)\exp(i\omega_{n,m}t)+\bar{z}_{n,m}(\tau)\exp(-i\omega_{n,m}t)\right]e_{n,m}(k,j),
\end{equation}
причем $z_{n,m}$ \--- пока неизвестные подлежащие определению амплитуды.
 
В данной работе обратимся лишь к нерезонансному случаю, т.е. выберем параметры $\varsigma_1, \varsigma_2$ таким образом, чтобы исключить всевозможные резонансы третьего порядка между собственными частотами $\omega_{n,m}$, иначк говоря, исключим одновременное выполнение равенств вида $\omega_{n_0,m_0} = s_1\omega_{n_1,m_1} +  s_2\omega_{n_2,m_2} + s_3\omega_{n_3,m_3},\ n_0 = \pm n_1 \pm n_2 \pm n_3, m_0 = \pm m_1 \pm m_2 \pm m_3$ для любых наборов индексов  $n_k,m_k, k=0,\dots,3$, таких что $(s_1, s_2, s_3): |s_1|+|s_2|+|s_3| = 3$ и при любой расстановке знаков во втором и третьем соотношениях. Тождественные резонансы $\omega_{n_0,m_0}=\omega_{n_0,m_0}+\omega_{n_1,m_1}-\omega_{n_1,m_1}$ из рассмотрения, естесственно, исключаются.

После подстановки (1) в краевую задачу (12) \-- (7) приравниваем коэффициенты при одинаковых степенях $\varepsilon$, после чего на первом шаге получаем верное тождество, а на втором приходим к следующей краевой задаче для $u_1(t,\tau,k,j):$
\begin{multline}
\frac{\partial^2u_1}{\partial t^2}+u_1-\mathcal{L}^{*}u_1 = \frac{\partial}{\partial t}\left[u_0+\nu \mathcal{L}^{*}u_0-\frac{u_0^3}{3}-2\frac{\partial u_0}{\partial\tau}\right],\\
u_1(0,j)=u_1(1,j), u_1(4,j)=u_1(5,j), u_1(k,0)=u_1(k,1), u_1(k,4)=u_1(k,5), \\
k=1,\dots,4, j=1,\dots,4.
\end{multline}
Система (3) разрешива в классе периодических по $t$ функций в том и только в том случае, если в неоднородности отсутствуют гармоники вида (13). Приравнивая коэффициенты при этих гармониках к нулю, приходим к системе относительно неизветсных комплексных амплитуд $z_{n,m}$
\begin{multline}
2\frac{dz{n,m}}{d\tau}=(1-\nu(\omega_{n,m}^2-1))z_{n,m}-\frac94 z_{n,m}|z_{n,m}|^2 - 3\sum_{k=0,k\not=n}^{3}z_{n,m}|z_{k,m}|^2 - \\
-3\sum_{s=0,s\not=m}^{3}z_{n,m}|z_{n,s}|^2-2\sum_{k,s=0,k\not=n,s\not=m}^{3}z_{n,m}|z_{k,s}|^2, \ \ \ n=0,\dots,3, \ m=0,\dots,3.
\end{multline}
Воспользовавшись заменой $z_{n,m}=\rho_{n,m}\exp(i\varPhi_{n,m})$, приходим к следующей системе, с которой и будем иметь дело в дальнейшем:
\begin{multline}
\frac{d\rho_{n,m}}{d\tau}=(1-\nu(\omega_{n,m}^2-1))\rho_{n,m}-\frac94 \rho_{n,m}^3 - 3\rho_{n,m}\sum_{k=0,k\not=n}^{3}\rho_{k,m}^2 - \\
-3\rho_{n,m}\sum_{s=0,s\not=m}^{3}\rho_{n,s}^2-2\rho_{n,m}\sum_{k,s=0,k\not=n,s\not=m}^{3}\rho_{k,s}^2,\ \ \ n=0,\dots,3, \ m=0,\dots,3.
\end{multline}
\end{document}
