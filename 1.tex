% !TeX encoding = UTF-8


\documentclass[12pt]{article}  %  Класса article хватит для курсовой и диплома 

\usepackage[utf8x]{inputenc}   % Подключение пакета inputenc и задание кодировки
\usepackage[english, russian]{babel}    % Локализация 
\usepackage{amsmath}		   % Пакет специальных символов
\usepackage{amsfonts}		   
\usepackage{amssymb}		   % Этих пакетов хватает для набора большинства спецсимволов	
\usepackage{graphicx}          % Графика
\usepackage{hyperref} 		   % Навигация по ссылкам в документе
\usepackage{cite} 		       % Продвинутое цитирование
\usepackage[a4paper, top=25mm, left=30mm, right=10mm, bottom=25mm]{geometry} % Поля 








\begin{document}         

Всюду далее будем считать, что, как и в [4], [5], $N_1 = N_2 = 4$. 
Для исследования локальной динамикии системы (12) \-- (7) в окрестности нулевого решения при фиксированном $\nu > 0$ и достаточн малом $0 < \varepsilon \ll 1$ выполним стандартную замену метода нормальных форм (см. [9])

\begin{equation}
u_{k,j}(t,\tau,\varepsilon) = u_0(t,\tau,k,j) + \varepsilon u_1(t,\tau,k,j) + \dots,\ \ \ \tau = \varepsilon t, \ \ \ k,j = 1,\dots,4.
\end{equation}
Здесь
\begin{equation}
u_0(t,\tau,k,j) = \sum_{n,m=1}^{4}\left[z_{n,m}(\tau)\exp(i\omega_{n,m}t)+\bar{z}_{n,m}(\tau)\exp(-i\omega_{n,m}t)\right]e_{n,m}(k,j),
\end{equation}
причем $z_{n,m}$ \--- пока неизвестные подлежащие определению амплитуды.
 
\end{document}
