% !TeX encoding = UTF-8


\documentclass[12pt]{article}  %  Класса article хватит для курсовой и диплома 

\usepackage[utf8x]{inputenc}   % Подключение пакета inputenc и задание кодировки
\usepackage[english, russian]{babel}    % Локализация 
\usepackage{amsmath}		   % Пакет специальных символов
\usepackage{amsfonts}		   
\usepackage{amssymb}		   % Этих пакетов хватает для набора большинства спецсимволов	
\usepackage{graphicx}          % Графика
\usepackage{hyperref} 		   % Навигация по ссылкам в документе
\usepackage{cite} 		       % Продвинутое цитирование
\usepackage[a4paper, top=25mm, left=30mm, right=10mm, bottom=25mm]{geometry} % Поля 

\begin{document}       
	
	Случай $m=4$ несколько более сложный. Выпишем значения ненулевых компонентов такого состояния равновесия
\begin{equation}
\begin{split}
&\rho^2_{p_1,q_1}=\frac{1}{33}(100(1-\nu(\omega^2_{p_1,q_1}-1))-32(1-\nu(\omega^2_{p_2,q_2}-1))-\\
&-32(1-\nu(\omega^2_{p_3,q_3}-1))-32(1-\nu(\omega^2_{p_4,q_4}-1))),\\
&\rho^2_{p_2,q_2}=\frac{1}{33}(100(1-\nu(\omega^2_{p_2,q_2}-1))-32(1-\nu(\omega^2_{p_1,q_1}-1))-\\
&-32(1-\nu(\omega^2_{p_3,q_3}-1))-32(1-\nu(\omega^2_{p_4,q_4}-1))),\\
&\rho^2_{p_3,q_3}=\frac{1}{33}(100(1-\nu(\omega^2_{p_3,q_3}-1))-32(1-\nu(\omega^2_{p_2,q_2}-1))-\\
&-32(1-\nu(\omega^2_{p_1,q_1}-1))-32(1-\nu(\omega^2_{p_4,q_4}-1))),\\
&\rho^2_{p_4,q_4}=\frac{1}{33}(100(1-\nu(\omega^2_{p_4,q_4}-1))-32(1-\nu(\omega^2_{p_2,q_2}-1))-\\
&-32(1-\nu(\omega^2_{p_3,q_3}-1))-32(1-\nu(\omega^2_{p_1,q_1}-1))).
\end{split}
\end{equation}
Понятно, что условия существования таких состояний равновесия имеют вид
\begin{equation}
\begin{split}
&100(1-\nu(\omega^2_{p_1,q_1}-1))-32(1-\nu(\omega^2_{p_2,q_2}-1))-\\
&-32(1-\nu(\omega^2_{p_3,q_3}-1))-32(1-\nu(\omega^2_{p_4,q_4}-1)) > 0, \\
&100(1-\nu(\omega^2_{p_2,q_2}-1))-32(1-\nu(\omega^2_{p_1,q_1}-1))-\\
&-32(1-\nu(\omega^2_{p_3,q_3}-1))-32(1-\nu(\omega^2_{p_4,q_4}-1)) > 0, \\
&100(1-\nu(\omega^2_{p_3,q_3}-1))-32(1-\nu(\omega^2_{p_2,q_2}-1))-\\
&-32(1-\nu(\omega^2_{p_1,q_1}-1))-32(1-\nu(\omega^2_{p_4,q_4}-1)) > 0, \\
&100(1-\nu(\omega^2_{p_4,q_4}-1))-32(1-\nu(\omega^2_{p_2,q_2}-1))-\\
&-32(1-\nu(\omega^2_{p_3,q_3}-1))-32(1-\nu(\omega^2_{p_1,q_1}-1)) > 0, \\
\end{split}
\end{equation}
Задача об устойчивости данных состояний равновесия распадается на две независящие друг от друга части. Коэффициенты первой из них имеют вид
\begin{equation}
\begin{split}
&\alpha_{(p_1,q_1),(p_1,q_1)} = -\frac{9}{2}\rho^2_{p_1,q_1}, \ \ \alpha_{(p_2,q_2),(p_2,q_2)} = -\frac{9}{2}\rho^2_{p_2,q_2},\\
&\dots,\\
&\alpha_{(p_1,q_1),(p_2,q_2)} = - 2C_{(p_1,q_1)(p_2,q_2)}\rho^2_{p_1,q_1}\rho^2_{p_2,q_2}, \ \ \alpha_{(p_2,q_2),(p_1,q_1)} = - 2C_{(p_2,q_2)(p_1,q_1)}\rho^2_{p_2,q_2}\rho^2_{p_1,q_1},\\
&\dots
\end{split}
\end{equation}
а второй \---
\begin{equation}
\begin{split}
&\alpha_{(i,j),(i,j)} = (1-\nu(\omega^2_{i,j}-1)) - \sum_{s=1}^{m}C_{(p_s,q_s)(i,j)}\rho^2_{(p_s,q_s)}, \\
&i\not=p_s \bigvee j\not=q_s, s=1,\dots,4,\\
&\alpha_{(i,j),(k,l)}=0,\\
&i\not=p_s \bigvee j\not=q_s, s=1,\dots,4, \ \ i\not=k, j\not=l\\
&i=0,\dots,3,j=0,\dots,3.
\end{split}
\end{equation}
\end{document}
